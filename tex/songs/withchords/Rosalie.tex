\beginsong{Rosalie}[by={Auteur}]
% Si vous souhaitez écrire les notes en (do, ré, mi ...) enlever le '%' dans la ligne ci-dessous.
\JecrisDoReMi

\transpose{\shift} % NE PAS EFFACER CETTE LIGNE.

% Qui ajoute cette chanson ici : Glo
% Lien vers une vidéo : 
% Comment j'ai obtenu les paroles: paroles.webfenua.com

% Raccourci pour les notes : (copiez les notes dans vos paroles)
% \[C]  \[D]  \[E]  \[F]  \[G]   \[A]  \[B] 
% \[DO] \[RE] \[MI] \[FA] \[SOL] \[LA] \[SI]
% E Bémol : \[E&]

% Pour l'orthographe en Tahitien, vérifiez l'orthographe sur: http://www.farevanaa.pf/dictionnaire.php
% voyelles avec tarava: \=a \=e \={\i} \=o \=u
%

\beginchorus
\[DO] Rosalie, elle est par\[SOL]tie
Si tu la voies, ramènes la \[DO]moi
Rosalie elle est par\[SOL]tie  
Si tu la voies, prends la pour  \[DO]toi
\rep{2}
\endchorus

\beginverse
\[Do] Elle est partie tout en claquant la \[SOL]porte  
Elle est partie que le diable l'em\[DO]porte 
Courir  ap\[FA] rès  ce n'est pas mon mé\[DO]tier
C'est bon pour \[REm] les chiens du quar\[SOL]tier  
J'espère un \[DO]jour apprendre qu'elle est \[SOL]morte
Qu'un gros camion l'a réduite en bouil\[C]lie
 Mon émo\[FA]tion  ne sera pas trop \[DO]forte 
Elle est par\[SOL]tie tant pis, tant pis, tant \[DO]pis  \[SOL]
\endverse

\beginchorus
\[DO] Rosalie, elle est par\[SOL]tie
Si tu la voies, ramènes la \[DO]moi
Rosalie elle est par\[SOL]tie  
Si tu la voies, prends la pour \[DO]toi
\endchorus


\endsong


