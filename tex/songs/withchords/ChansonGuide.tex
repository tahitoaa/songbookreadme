\beginsong{Titre}[by={Auteur}]
% Si vous souhaitez écrire les notes en (do, ré, mi ...) enlever le '%' dans la ligne ci-dessous.
% \JecrisDoReMi
\transpose{\shift} % NE PAS EFFACER CETTE LIGNE.

% Qui ajoute cette chanson ici : mon pseudo
% Lien vers une vidéo : 
% Comment j'ai obtenu les paroles: 

% Raccourci pour les notes : (copiez les notes dans vos paroles)
% \[C]  \[D]  \[E]  \[F]  \[G]   \[A]  \[B] 
% \[DO] \[RE] \[MI] \[FA] \[SOL] \[LA] \[SI]
% E Bémol : \[E&]

% Pour l'orthographe en Tahitien, vérifiez l'orthographe sur: http://www.farevanaa.pf/dictionnaire.php
% voyelles avec tarava: \=a \=e \={\i} \=o \=u
%

% Pour les intros/bridge/ending (tout ce qui n'est pas un verse ou chorus, mettre comme suit:

% \beginverse*
% Coucou
% \endverse

\beginverse
\[C] Marigni nei toku roima\[Am]ta e
No toku i\[F]nagnaro ia koe mai\[C]ne
To mata ma\[G]nea mai te pura\[Am]gna e
E ata ma\[F]nea ki rugna ia koe e mai\[c]ne
\endverse

\beginchorus
\[G]Manako manako manako ia ko\[Am]e
Maupiri iau \[F]ia ko\[C]e \rep{2}
To koe ana\[G]ke anake i\[Am]e
Maupiri \[F]iau ia ko\[C]e 
\rep{2}
\endchorus

\endsong