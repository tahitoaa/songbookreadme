
\newpage
\maketitle
\section*{Introduction}

\noindent Les explications pour particier à ce répertoire sont dans ce \href{https://github.com/tahitoaa/songbookreadme/blob/master/songbookreadme.md}{Readme}.

\noindent Pour éditer ce pdf vous devez créer un compte sur overleaf en suivant \href{https://fr.overleaf.com/9457921969mhjkwjdnycqj}{ce lien}.

\noindent Il est possible de \emph{transposer} une, plusieurs, ou toutes les chansons de ce répertoire. Lire le \href{https://github.com/tahitoaa/songbookreadme/blob/master/songbookreadme.md}{Readme} pour en savoir plus.

% Il est possible de transposer toutes les chansons du répertoire en une fois. Pour cela il faut modifier la valeur du \texttt{\\shift} avec la commande \texttt{\\renewcommand{\\shift}{<n>}} en remplacant \texttt{<n>} par le nombre de notes à transposer (en comptant les demi-tons).

\subsection*{Liens Utiles}

\begin{enumerate}
    \item \href{http://tahitiansongs.fr/wp-content/uploads/2011/11/carnet_de_chants2.pdf}{Le carnet de chant tahitiansongs.fr} contient une centaine de chansons classiques, et au même format que ce répertoire.
    \item \href{http://paroles.webfenua.com/list.php}{Le site paroles.webfenua.com} contient \emph{beaucoup} de chanson, il est \emph{très complet} et régulièrement mis à jour (4786 chansons).
\end{enumerate}

\newpage

%\showindex[1]{Index of Song Titles}{titlidx}
%\showindex[2]{Index of Song Authors}{authidx}
\newpage
